\documentclass[twocolumn,DIV19]{scrartcl}
\usepackage[utf8]{inputenc}
%\usepackage[T1]{fontenc}
\usepackage{graphicx}
\usepackage{longtable}
\usepackage{float}
\usepackage{wrapfig}
\usepackage{soul}
\usepackage{amssymb}
\usepackage{amsmath}
\usepackage{hyperref}
\usepackage{color}
\usepackage{units}

\title{Lisp and the Open Graphics Development Board}
\author{Martin Kielhorn}
\date{2011-06-29}

\begin{document}
\maketitle 

Right now I am supposed to write up my PhD but for some reason I
decided to venture into the world of hardware design and low level
programming.

During the last few days I spent some time trying to understand the
open graphics board OGD1. I suspected it would be interesting but I
wasn't aware that it would be so much fun.

I do my PhD in physics. More precisely I work in the field of
optics. We improve a fluorescent microscope. Normally these
microscopes shine blue light into a sample. The sample contains
fluorophores that will emit green light upon exposure. This green
light is then detected with a camera. Instead of sending as much light
into the specimen as possible (this is what normal microscopes do) we
try to control exactly from which directions and what parts of the
sample are illuminated.



\end{document}
