\documentclass[%twocolumn,
  DIV19]{scrartcl}
\usepackage[utf8]{inputenc}
%\usepackage[T1]{fontenc}
\usepackage{graphicx}
\usepackage{longtable}
\usepackage{float}
\usepackage{wrapfig}
\usepackage{soul}
\usepackage{amssymb}
\usepackage{amsmath}
\usepackage{hyperref}
\usepackage{color}
\usepackage{units}
\newcommand{\bild}[1]{\includegraphics[width=14cm]{#1}}
\usepackage{listings}
  \usepackage{courier}
 \lstset{
         basicstyle=\footnotesize\ttfamily, % Standardschrift
         %numbers=left,               % Ort der Zeilennummern
         numberstyle=\tiny,          % Stil der Zeilennummern
         %stepnumber=2,               % Abstand zwischen den Zeilennummern
         numbersep=5pt,              % Abstand der Nummern zum Text
         tabsize=2,                  % Groesse von Tabs
         extendedchars=true,         %
         breaklines=true,            % Zeilen werden Umgebrochen
         keywordstyle=\color{red},
    		frame=b,         
 %        keywordstyle=[1]\textbf,    % Stil der Keywords
 %        keywordstyle=[2]\textbf,    %
 %        keywordstyle=[3]\textbf,    %
 %        keywordstyle=[4]\textbf,   \sqrt{\sqrt{}} %
         stringstyle=\color{white}\ttfamily, % Farbe der String
         showspaces=false,           % Leerzeichen anzeigen ?
         showtabs=false,             % Tabs anzeigen ?
         xleftmargin=17pt,
         framexleftmargin=17pt,
         framexrightmargin=5pt,
         framexbottommargin=4pt,
         %backgroundcolor=\color{lightgray},
         showstringspaces=false      % Leerzeichen in Strings anzeigen ?        
 }
 \lstloadlanguages{% Check Dokumentation for further languages ...
         %[Visual]Basic
         %Pascal
         %C
         %C++
         %XML
         %HTML
   %Java
   Lisp
 }
    %\DeclareCaptionFont{blue}{\color{blue}} 

  %\captionsetup[lstlisting]{singlelinecheck=false, labelfont={blue}, textfont={blue}}
  \usepackage{caption}
\DeclareCaptionFont{white}{\color{white}}
\DeclareCaptionFormat{listing}{\colorbox[cmyk]{0.43, 0.35, 0.35,0.01}{\parbox{\textwidth}{\hspace{15pt}#1#2#3}}}
\captionsetup[lstlisting]{format=listing,labelfont=white,textfont=white, singlelinecheck=false, margin=0pt, font={bf,footnotesize}}


\title{Lisp and the Open Graphics Development Board}
\author{Martin Kielhorn (kielhorn.martin@googlemail.com)}
\date{2011-06-29}

\begin{document}
\maketitle 

\section{Somewhat long introduction}
Right now I am supposed to write up my PhD but for some reason I
decided to venture into the world of hardware design and low level
programming.

During the last few days I spent some time trying to understand the
open graphics board OGD1. I suspected it would be interesting but I
wasn't aware that it would be so much fun.

I do my PhD in physics. More precisely, I work in the field of
optics. We improve a fluorescent microscope. Normally these
microscopes shine blue light into a sample. The sample contains
fluorophores that will emit green light upon exposure. This green
light is then detected with a camera. Instead of sending as much light
into the specimen as possible (this is what normal microscopes do) we
try to control exactly from which directions and what parts of the
sample are illuminated. This means specimen don't die as fast as they
usually do (we look at \emph{C. elegans}). I hope that we can
eventually use this system for optogenetic research to characterize
neuronal networks.

\begin{figure}
  \centering
  \bild{system.jpg}
  \caption{Our microscope with the new illumination system.}
\end{figure}

You could say we use a data projector to shine light into the
microscope. As opposed to a normal data projector we actually use two
displays and as a light source we use a laser.

But one of these displays is connected to a graphics card with a
digital DVI output. For various reasons the displays can only run
below 50\% duty cycle and need to be synchronized correctly.

I have to say that it is quite an irritating experience, trying to
synchronize either an NVIDIA or ATI graphics card (with the propietary
drivers) to the vertical blank. I tried to use several OpenGL
extensions (\verb!GLX_SGI_swap_control!, \verb!GLX_SGI_video_syn!)
that should make this synchronization possible. But I didn't have any
success. I used \verb!nm! and \verb!objdump -CDx! on libgl.so in order
to figure out which of the functions are stubs and which actually
contain some sensible code. However, I only managed to create some
zombie processes.

\begin{figure}
  \centering
  \bild{screen_numbers.png}
  \caption{In one test the graphics card (NVIDIA in this case)
    displays each frame's number. This picture is then projected on a
    fluorescent plane and captured by the camera. If sync-to-vblank
    isn't working or the camera is too slow the pictures of the camera
    will be distorted. }
\end{figure}


Earlier in the project I bought one OGD1 PCI card. When it arrived I
was a bit confused because the card comes in PCI-X format. It looks a
lot bigger than the usual PCI card. Incidentally our only Xeon
computer that did have such a socket broke. So I thought, I could
never use this card. However, a few days ago I learned that one can
use it in any PCI socket.

\section{Getting to work}
There is a Wiki for the project at
\verb!http://wiki.opengraphics.org/tiki-index.php!. This web page is a
bit convoluted. I found it useful to filter out the more recent
entries using Google's 'More Search Tools', especially 'Past Month'.

Furthermore there are very helpful, fast responding people on the
projects mailing list.

The first thing you should do, is to look through their code:
\verb!svn co svn://svn.opengraphics.org/ogp!.

When I first received it, I was a bit overwhelmed but the directory
structure gets comprehensible when you get accustomed to the projects
nomenclature.

\subsection{rtl/ Hardware description}
Up to now I had little exposure to hardware description languages. The
OGP1 board contains two FPGAs. The lattice XP10 to interface to the
PCI bus and a Xilinx Spartan 3.

I opened the the project in a the Xilinx software, which you can
download free of charge. However I didn't succeed in building an
image, but then I didn't try very hard. I have to say that I find the
11 gigabytes Xilinx toolchain quite repulsive.

Anyway, the file \verb!rtl/include/registers.txt! is very
important. It contains the definitions of the registers that are
needed to communicate with the hardware and is the file I spent the
most time looking at.

\subsection{drivers/ Standard interfaces}
The first thing I did after installing the hardware, was to test it.
I did this by compiling the kernel modules for the framebuffer driver.
Unfortunately I didn't manage to compile the Xwindow driver because
the autotools of Debian 5 don't seem to be recent enough.

The good thing about these drivers are that they are in an early stage
of development and quite easy to understand. The only task the kernel
module \verb!ogp_skel! accomplishes is to register a handler for
interrupts. Fortunately this is exactly what I'm interested in: I want
to receive interrupts with the vertical refresh of the screen.

\subsection{tools/oga1lib/ User level control}
I learned most about the system by going through
\verb!tools/oga1utils/bin/oga1-vid-test.c!. This program intializes a
video mode and draws an animation pixel by pixel to the screen.

The graphics card communicates via I2C with the computer screen in
order to find out the ideal pixel clock rate and resolution (see
parse-ddc). In my case it turned out that there were some errors in
the way the clock dividers were set which resulted in a frame rate of
\unit[47]{Hz} instead of \unit[60]{Hz}. In order to understand what
was going on I first wrote an sb-alien foreign function interface to
call \verb!liboga1.so! and then gradually replaced the higher level
functions with my own lisp functions.

During all this time the graphics card or my SBCL image never crashed.
I think this is when I realized how much fun this is. You can stay in
user level and experiment with all kinds of things. You can turn the
DVI output on and off, you can connect different screens and ask for
their EDID information -- and you never have to restart the
computer. Actually right now I am logged into the lab computer from
home via and work remotely via SLIME and Tramp.

The last thing I worked on is the assembler for the instruction of the
video controller. The video controller is implemented in the Xilinx
FPGA and is in charge of sending the data from the graphics memory
through the cable. It prepends the frame with a VSYNC pulse and each
line with a HSYNC pulse (see the picture with \verb!HSYNC->! in the
Lisp listing). This controller is also in charge of notifying the host
CPU when a frame is finished by generating an interrupt that is then
routed through the XP10 onto the PCI bus.

\subsection{Interrupts}
I'm currently trying to figure out, how I can enable the interrupt
reporting and receive them on the CPU. Ideally one would use a small
\verb!uio! kernel module that would allow to report the IRQs to user
level.


\section{Outlook}
Other things that I want to look at are iverilog and
gtkwave. Apparently one can simulate the hardware and look at the
waveforms that the FPGA would generate.

\begin{figure}
  \centering
  \bild{screen_gtkwave.png}
  \caption{Screenshot of gtkwave with simulated waveform.}
\end{figure}

\section{Appendix Listing}
\begin{lstlisting}[label=some-code,caption=SBCL Code to talk to graphics card]
(defpackage :oga1-ffi
  (:use :cl :sb-alien :sb-c-call))
(in-package :oga1-ffi)

(load-shared-object "/opt/ogp/lib/liboga1.so")

(defparameter *ddc* 0)
(defparameter *card* 0)

(defmacro defconstants (&rest rest)
  "Define multiple constants. Reduces the amount that needs to be
written."
  `(progn
     ,@(loop for e in rest collect
	    (destructuring-bind (name val) e
	      `(defconstant ,(intern (format nil "+~A+" name))
		 ,val)))))

(defconstants
    (vmem-reg 0)
    (vmem-mem 1)
  (head-top 0)
  (head-bottom 1)
  (upload-pbrom 0)
  (upload-cprom 1)
  (upload-hqprog 2)
  (upload-hqmem 3)
  (upload-vm0 4)
  (upload-vm1 5)
  ;; i2c
  (ddc-top 0)
  (ddc-bottom 1)
  (i2c-top 2)
  (i2c-bottom 3)
  ;; constants from reg_defines
  ;; block xp10
  (xp10-video-reset #x20)
  (xp10-interrupt-status #x24)
  (xp10-interrupt-clear #x24)
  (xp10-dac-power-on #x28)
  (xp10-interrupt-mask #x30) ;; rw
  (xp10-test-reg #x7c) ;; rw
  ;; block s3
  (s3-pll-min 180000000)
  (s3-pll-max 360000000)
  (s3-ref-clock0 133330000)
  (s3-ref-clock1 125000000)
  ;; block vm
  (vm0-program-memory #x3000)
  (vm1-program-memory #x4000)
  ;; block vc
  (vc0-pixel-x-start #x3800)
  (vc0-pixel-y-start #x3804)
  (vc0-clear-int #x3808)
  (vc0-pixel-info #x380c) 
  (vc0-cpu-reset-b #x3810)
  (vc0-interrupt-enable #x3814)
  (vc0-video-enable #x3818)
  (vc0-pc-start #x3820)
  (vc0-dividers #x3824)
  (vc0-output-mode #x3828)
  ;; dividers
  (dividers-divisor1-mask 7)
  (dividers-divisor1-lsb-posn 0)
  (dividers-divisor1-msb-posn 2)
  (dividers-divisor0-mask #x1f8)
  (dividers-divisor0-lsb-posn 3)
  (dividers-divisor0-msb-posn 8)
  (dividers-base-clock-mask #x200)
  (dividers-base-clock-posn 9)
  (dividers-oe-mask #x400)
  (dividers-oe-posn 10)
  ;;
  (vid-fb-offset-t #x8000000)
  )

(define-alien-type u8 unsigned-char)
(define-alien-type u32 unsigned-int)
(define-alien-type card (* int))
(define-alien-type upload int)
(define-alien-type i2c int)
(define-alien-type head int)
(define-alien-type nil
    (struct mode-info
	    (pixel-clock u32)
	    (hres u32) (hfp u32) (hsync u32) (hblank u32)
	    (vres u32) (vfp u32) (vsync u32) (vblank u32)
	    (digital u8)
	    (hsync-is-low u8)
	    (vsync-is-low u8)
	    (pix-depth u8)
	    (fb-start u32)
	    (pitch u32)))
(define-alien-type mode-info-ptr (* (struct mode-info)))
(define-alien-type size_t unsigned-long)

(eval-when (:compile-toplevel)
  (defun replace-_ (str)
    "Replaces hyphens with underscores. To convert Lisp names into C
names."
    (dotimes (i (length str) str)
     (when (eq (char str i) #\-)
       (setf (char str i) #\_)))))

(defmacro defc (name ret args &key (card t))
  "Define a foreign function of the liboga1* libraries. Most of them
get card as first parameter. If CARD is nil no such parameter is
defined."
  (progn
    (when card
      (push '(card card) args))
   `(define-alien-routine (,(string-downcase 
			     (replace-_ (format nil "oga1_~a" name)))
			    ,(intern (format nil "~a" name)))
	,ret
      ,@args)))
 


(defc card-rset void ((reg u32) (value u32)))
(defc card-mset void ((reg u32) (value u32)))
(defc card-rget u32 ((reg u32)))
(defc card-mget u32 ((reg u32)))
(defc card-mcopy void ((addr u32) (data (* u8)) (length u32)))

(defc pci-read-long u32 ((addr u32)))
(defc pci-write-long void ((addr u32) (val u32)))
(defc card-rset-config void ((reg u32) (val u32)))
(defc card-rget-config u32 ((addr u32)))
(defc card-mset-config void ((addr u32) (val u32)))

(defc card-erase void ((where upload)))
(defc card-upload void ((where upload) (base u32) (vbase (* int)) 
			(size size_t)))

(defc card-hq-start void ((bypass int)))
(defc card-hq-stop void ())

(defc card-hq-dmem-set void ((base u32) (data u32)))
(defc card-hq-dmem-get u32 ((base u32)))
(defc card-hq-dmem-set-config void ((base u32) (data u32)))
(defc card-hq-dmem-get-config u32 ((base u32)))

(defc card-mem-init void ())

(defc get-clock-dividers u32 ((freq u32)))
(defc get-clock-frequency u32 ((dividers u32)))

(defc video-reset void ((top-digital int)))

(defc set-mode void ((head head) (inf mode-info-ptr) (program (* u32))))

(defc set-video-clock void ((head head) (digital int) (pixel-clock u32)))

(defc load-video-program void ((head head) (inf mode-info-ptr) 
			       (program (* u32))))

(defc set-video-mode void ((head head) (digital int) (pixel-clock u32)
			   (pixel-depth u32)))

(defc dvi-init void ((head head) (pixel-clock u32)))

(defc dvi-init-single-link void ((head head) (pixel-clock u32)))

(defc dvi-init-dual-link void ((head head) (pixel-clock u32)))

(defc edid-find-video-mode int
  ((info mode-info-ptr)
   (width u32) (height u32) (refresh u32)
   (ddc (* u8))))

(defc i2c-init void ((i2c i2c)))
(defc i2c-write2 int ((i2c i2c) (addr u8) (b0 u8) (b1 u8)))
(defc i2c-get-edid int ((head head) (ddc (* u8))))

;; user
(load-shared-object "/opt/ogp/lib/liboga1-user.so")

(defc card-open card ((bus int) (dev int) (func int)) :card nil)
(defc card-free void nil)
(defc card-set-log-file void ((file-pointer (* int))))
(defc edid-dump void ((file-pointer (* int))
		      (ddc (* u8))))

(defun parse-ddc ()
 (macrolet ((q (i)
	      `(aref *ddc* ,i))
	    (active (i)
	      `(+ (* 256 (ldb (byte 4 4) (q ,(+ 2 i)))) (q ,i)))
	    (blank (i)
	      `(+ (* 256 (ldb (byte 4 0) (q ,(1+ i)))) (q ,i))))
   (list :pixel-clock-hz (* 10000 (+ (* 256 (q 55))
				     (q 54)))
	 :hact (active 56)
	 :hblank (blank 57)
	 :vact (active 59)
	 :vblank (blank 60)
	 :hsync-off (+ (* 256 (ldb (byte 2 6) (q 65))) (q 62))
	 :hsync-width (+ (* 256 (ldb (byte 2 4) (q 65))) (q 63))
	 :vsync-off (+ (* 16 (ldb (byte 2 2) (q 65))) (ldb (byte 4 4) (q 64)))
	 :vsync-width (+ (* 16 (ldb (byte 2 0) (q 65))) (ldb (byte 4 0) (q 64)))
	 :xsize-mm (+ (* 256 (ldb (byte 4 4) (q 68))) (q 66))
	 :ysize-mm (+ (* 256 (ldb (byte 4 0) (q 68))) (q 67))
	 :hborder (q 69)
	 :vborder (q 70)
	 :interlaced (ldb (byte 1 7) (q 71))
	 :stereo (ldb (byte 2 5) (q 71))
	 :separate-sync (ldb (byte 2 3) (q 71))
	 :vsync-positive (ldb (byte 1 2) (q 71))
	 :hsync-positive (ldb (byte 1 1) (q 71))
	 :stereo (ldb (byte 1 0) (q 71)))))

(defun dividers-to-u32 (oe base-clock divisor0 divisor1)
  "Combine parameters into a value that can be written to the dividers
register."
  (declare (type (unsigned-byte 1) oe base-clock)
	   (type (unsigned-byte 3) divisor1) 
	   (type (unsigned-byte 6) divisor0) 
	   (values (unsigned-byte 32) &optional))
  (let ((res 0))
    (setf (ldb (byte 1 +dividers-oe-posn+) res) oe
	  (ldb (byte 1 +dividers-base-clock-posn+) res) base-clock
	  (ldb (byte 3 +dividers-divisor1-lsb-posn+) res) divisor1
	  (ldb (byte 6 +dividers-divisor0-lsb-posn+) res) divisor0)
    res))

(defun dividers-u32-to-list (v)
  "Parse the value from the dividers register into its components."
  (declare (type (unsigned-byte 32) v))
  (list :oe (ldb (byte 1 +dividers-oe-posn+) v)
	:base-clock (ldb (byte 1 +dividers-base-clock-posn+) v)
	:divisor0 (ldb (byte 6 +dividers-divisor0-lsb-posn+) v)
	:divisor1 (ldb (byte 3 +dividers-divisor1-lsb-posn+) v)))

(defun get-divided-frequency (ref-clock pre post)
  (* 48 (/ ref-clock (ash pre 
			  (1- post)))))

(defun lisp-get-clock-dividers (freq)
  "Find the 4 best clock divider combinations to achieve frequency."
  (declare (type (unsigned-byte 32) freq))
  (let ((res nil))
    (let ((max-freq (+ freq (/ (* 5 freq) 1000))))
      (dotimes (sel 2)
	(let ((clk (elt (list +s3-ref-clock0+
			      +s3-ref-clock1+)
			sel)))
	  (loop for pre from 1 upto 63 do
	       (when (< +s3-pll-min+ (* 48 (/ clk pre)) +s3-pll-max+)
		 (loop for post from 1 upto 6 do
		      (let ((test-clk (get-divided-frequency clk pre post)))
			(when (<= test-clk max-freq)
			  (let ((diff (abs (- freq test-clk))))
			    (push (list diff (* 1d0 (/ diff test-clk)) sel pre post) res))))))))))
    (subseq (sort res #'< :key #'first)
	    0 4)))

(define-alien-routine progressive int
  (program (* u32))
  (fb-width u32)
  (fb-height u32)
  (vp-width u32)
  (vp-height u32)
  (depth-bpp u32)
  (hfp u32)
  (hsync u32)
  (hbp u32)
  (vfp u32)
  (vsync u32)
  (vbp u32)
  (viewport u32)
  (pixperclock u32)
  (horiz-assert-low u32)
  (vert-assert-low u32))

(defun lisp-vm-upload (base program n)
  (declare (type (unsigned-byte 32) base n)
	   (type (simple-array (unsigned-byte 32) (512)) program))
 (let ((addr +vm0-program-memory+))
   (incf addr (* 4 base))
   (dotimes (i n)
     (card-rset *card* addr (aref program i))
     (incf addr 4))))

(defun lisp-set-video-clock ()
  (card-rset *card* +vc0-dividers+ (dividers-to-u32 0 1 28 1))
  (sleep .001)
  (card-rset *card* +vc0-dividers+ (dividers-to-u32 1 1 28 1))
  (sleep .001))

(defun lisp-set-pixel-info (depth rate)
  (declare (type (unsigned-byte 3) depth)
	   (type (unsigned-byte 1) rate))
  (let ((v 0))
    (declare (type (unsigned-byte 32) v))
    (setf (ldb (byte 1 3) v) rate
	  (ldb (byte 3 0) v) depth)
    (card-rset *card* +vc0-pixel-info+ v)
    v))

(defun lisp-set-output-mode (digital not-dvi-sl dac-de-en dac-sync-en
			     vsync-polarity hsync-polarity)
  (declare (type (unsigned-byte 1) digital not-dvi-sl dac-de-en dac-sync-en
		 vsync-polarity hsync-polarity))
  (let ((v 0))
    (declare (type (unsigned-byte 32) v))
    (setf (ldb (byte 1 0) v) digital
	  (ldb (byte 1 1) v) not-dvi-sl
	  (ldb (byte 1 2) v) dac-de-en
	  (ldb (byte 1 3) v) dac-sync-en
	  (ldb (byte 1 4) v) vsync-polarity
	  (ldb (byte 1 5) v) hsync-polarity)
    (card-rset *card* +vc0-output-mode+ v)
    v))

(defun lisp-unset-video-mode ()
  (card-rset *card* +vc0-cpu-reset-b+ 0)  ;; disable video program counter
  (card-rset *card* +vc0-video-enable+ 0) ;; disable video controller
  (card-rset *card* +vc0-pc-start+ 0)     ;; reset program counter
  (card-rset *card* +vc0-interrupt-enable+ 0))

(defun lisp-set-video-mode ()
  (lisp-set-output-mode 1 0 0 0 1 0)
  (sleep .001)
  (lisp-unset-video-mode)
  (let ((dvi-sl 1)
	(vid-mode 5))
    (lisp-set-pixel-info vid-mode dvi-sl))
  
  (lisp-set-output-mode 1 0 0 0 0 1)
  (card-rset *card* +vc0-cpu-reset-b+ 1)
  (card-rset *card* +vc0-video-enable+ 1)
  (card-rset *card* +vc0-clear-int+ 1)
  (card-rset *card* +xp10-interrupt-mask+ #b111)
  (sleep .001)
  (card-rset *card* +vc0-interrupt-enable+ 1))

#+nil
(card-rset *card* +xp10-interrupt-mask+ #b111)
#+nil
(card-rget *card* +xp10-interrupt-mask+)
#+nil
(card-rset *card* +vc0-interrupt-enable+ 1)
#+nil
(card-rset *card* +vc0-video-enable+ 1)
#+nil
(card-rset *card* +vc0-clear-int+ 1)
#+nil
(card-rget *card* +xp10-interrupt-status+)
#+nil
(progn
 (card-rset *card* +xp10-interrupt-clear+ 0)
 (card-rget *card* +xp10-interrupt-status+))


(defun lisp-set-mode ()
  (lisp-set-video-clock)
  (dvi-init *card* +head-top+ 108090000)

  (let* ((program (make-array 512 :element-type '(unsigned-byte 32)))
	 (sap (sb-sys:vector-sap program))
	 (n   (progressive sap
			   1280 1024
			   1280 1024
			   32 
			   48 112 248
			   1 3 38
			   #x80 2
			   0 0)))
    (lisp-vm-upload 0 program n)) 
  
  (lisp-set-video-mode))

#+nil
(defparameter *card* (card-open #x0b 1 0))
#+nil
(card-mem-init *card*)
#+nil
(let ((ddc (make-array 128 :element-type '(unsigned-byte 8))))
 (defparameter *top* (i2c-get-edid *card* +head-top+ (sb-sys:vector-sap ddc)))
 ;; 0 success, 1 timeout, 2 bad head, 3 bad header, 4 bad checksum
 (defparameter *ddc* ddc)
 (parse-ddc))

(defun lisp-video-reset ()
  (card-rset *card* +xp10-video-reset+ 0)
  (sleep .001)
  (card-rset *card* +xp10-video-reset+ 1)
  (card-rset *card* +xp10-dac-power-on+ 0))

#+nil ;; turn DAC off
(lisp-video-reset)

#+nil
(lisp-set-mode)

#+nil
(lisp-unset-video-mode)

#+nil
(card-mget *card* +vid-fb-offset-t+)


#+nil
(card-mset *card* +vid-fb-offset-t+ #x00123456)
#+nil
(card-mget *card* +vid-fb-offset-t+)

(defun draw-screen ()
  (let ((pitch 1280))
   (dotimes (i 1280)
     (dotimes (j 1024)
       (declare (type (unsigned-byte 16) j i pitch))
       (let ((addr (+ +vid-fb-offset-t+
		      (* 4 (+ (* pitch j)
			      i)))))
	 (card-mset *card* addr #xfaffaaff))))))
#+nil
(time
 (draw-screen))


(defmacro def-choice (name choices)
  "Define a function that maps a keyword into a number."
  `(defun ,name (&optional (param ,(first (first choices))))
     (declare (type (member ,@(mapcar #'first choices)) param))
     (ecase param ,@choices)))

(def-choice cursor-to-nr ((:no-change #b0)
			  (:advance-x #b01)
			  (:reset-x-advance-y #b10)
			  (:reset-xy #b11)))

(def-choice opcode-to-nr ((:jump 0)
			  (:call 1)
			  (:wait 2)
			  (:send 3)
			  (:fetch 5)
			  (:addr 6)
			  (:inc 7)))

(defun vc (opcode &key (de 0) (vsync 0) (hsync 0)
	   (cursor :no-change) (ret 0) (int 0) 
	   (count 0 count-p) (iaddr 0 iaddr-p) 
	   (memory-address 0 memory-address-p))
  "Construct an instruction for the video controller."
  (declare (type (unsigned-byte 1) de vsync hsync ret int)
	   (TYPE (MEMBER :NO-CHANGE :ADVANCE-X
			 :RESET-X-ADVANCE-Y :RESET-XY) cursor)
	   (TYPE (MEMBER :JUMP :CALL :WAIT :SEND 
			 :FETCH :ADDR :INC) opcode)
	   (type (unsigned-byte 12) count)
	   (type (unsigned-byte 9) iaddr)
	   (type (unsigned-byte 21) memory-address)
	   (values (unsigned-byte 32) &optional))
  (when (and count-p memory-address-p)
    (break "count and memory-address given at the same time."))
  (when (and iaddr-p memory-address-p)
    (break "iaddr and memory-address given at the same time."))
  (when (and (member opcode'(:addr :inc))
	     (or count-p iaddr-p))
    (break ":addr and :inc expect :memory-address instead of :count."))
  (let ((v 0)
	(opcode-nr (opcode-to-nr opcode))
	(cursor-nr (cursor-to-nr cursor)))
    (declare (type (unsigned-byte 3) opcode-nr)
	     (type (unsigned-byte 2) cursor-nr))
    (when (member opcode '(:call :wait :send))
      (setf count (- 2049 count)))
    (setf (ldb (byte 1 31) v) de
	  (ldb (byte 1 30) v) vsync 
	  (ldb (byte 1 29) v) hsync
	  (ldb (byte 2 27) v) cursor-nr
	  (ldb (byte 1 26) v) ret
	  (ldb (byte 1 25) v) int
	  (ldb (byte 3 21) v) opcode-nr)
    (when (or count-p iaddr-p)
      (setf (ldb (byte 12 9) v) count
	    (ldb (byte 9 0) v) iaddr))
    (when memory-address-p
      (setf (ldb (byte 21 0) v) memory-address))
    v))

;; HSYNC-> HBP->    VIDEO->           HFP->
;; --------................................      VFP
;; --------................................      VFP
;; ++++++++||||||||||||||||||||||||||||||||      VSYNC
;; ++++++++||||||||||||||||||||||||||||||||      VSYNC
;; --------................................      VBP
;; --------................................      VBP
;; --------........################........      VIDEO
;; --------........################........      VIDEO
;; --------........################........      VIDEO
;; --------........################........      VIDEO
;; --------........################........      VIDEO
;; Legend:
;;  - Horizontal sync asserted
;;  | Vertical sync asserted
;;  + Both Horizontal and Vertical sync asserted
;;  . Blanking
;;  # Active video


(progn 
  (defun lisp-progressive (&key 
			   (fb-width 1280) (fb-height 1024)
			   (vp-width fb-width) (vp-height fb-height)
			   (depth 32)
			   (hres 1280) (hfp 48) (hsync 112) (hbp 248)
			   (vres 1024) (vfp  1) (vsync   3) (vbp  38)
			   (viewport 0) 
			    (pixperclock 2))
    (list
     (vc :call :count vfp :iaddr 23 :hsync 1)
     (vc :call :count vsync :iaddr 27 :hsync 1 :vsync 1)
     (vc :call :count (1- vbp) :iaddr 23 :hsync 1 :vsync 1)
     (vc :addr :memory-address (/ viewport 128) :hsync 1 :cursor :reset-xy)
     (vc :wait :count (1- (/ hsync pixperclock)) :hsync 1)
     (vc :fetch :count (/ (* vres depth) 256)) 
     ;; there must be 3 instructions before next fetch, addr, inc
     (vc :wait :count (- (/ vp-width pixperclock) 3))
     (vc :inc )))
  (let ((l (lisp-progressive)))
    (loop for i below (length l) collect
	 (list i (elt l i)))))
\end{lstlisting}

\end{document}
